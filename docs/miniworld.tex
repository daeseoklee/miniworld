\documentclass{amsart}
\usepackage{geometry}
 \geometry{
 a4paper,
 total={170mm,257mm},
 left=20mm,
 top=20mm,
 }
\usepackage{amsmath,amsfonts,amssymb}
\usepackage{bm}
% Theorem environments-------------------------------------------
\newtheorem{thm}{Theorem}[section]
\newtheorem{cor}[thm]{Corollary}
\newtheorem{lem}[thm]{Lemma}
\newtheorem{prop}[thm]{Proposition}
\theoremstyle{definition}
\newtheorem{defn}[thm]{Definition}
\theoremstyle{remark}
\newtheorem{rem}[thm]{Remark}
\numberwithin{equation}{section}
% simple commands---------------------------------------------
\newcommand{\abs}[1]{\left\vert#1\right\vert}
\newcommand{\set}[1]{\left\{#1\right\}}
\newcommand{\seq}[1]{\left<#1\right>}
\newcommand{\norm}[1]{\left\Vert#1\right\Vert}
%additional-------------------------------------------------------
\newtheorem{prob}{Problem}[]
%------------------------------------------------------
\begin{document}
\title{miniworld core 0.2.0}
\date{\today}
\author{Daeseok Lee}
\begin{abstract}
This is do
\end{abstract}
\maketitle
%------------------------------------------------------
\section{introduction}
\subsection{miniworld project}


\subsection{miniworld core} 


\subsection{version information} 
\subsubsection{version 0.2.0}


%------------------------------------------------------
\section{ecosystem outline}
\subsection{terminology}


\subsection{outline}


\subsection{life cycle} 


\subsection{action}


\subsection{mineral dynamics}


\subsection{reproduction}  

%------------------------------------------------------
\section{free parameters}
\subsection{initial configuration}

\subsection{life cycle}

\subsection{action}

\subsection{mineral dynamics}

\subsection{reproduction}

%------------------------------------------------------
\section{public API}

\subsection{free parameter manipulation}

\subsection{neural network manipulation}

\subsection{world generation} 

\subsection{execution} 

\subsection{measurement}

\subsection{visualization helper}



%----------------------------------------------
\section{ecosystem detail} 
\subsection{life cycle}

\subsection{action}

\subsection{mineral dynamics}

\subsection{reproduction}
%------------------------------------------------------
\section{example simulation}

\end{document}







===============================================================
\documentclass[a4paper,11pt]{article}
\usepackage{graphicx,fancyhdr,amsmath,amssymb,amsthm,subfig,url,hyperref}
\usepackage{dhucs}
\usepackage[margin=1in]{geometry}
%-----------------------------------
\newcommand{\abs}[1]{\left\vert#1\right\vert}
\newcommand{\set}[1]{\left\{#1\right\}}
\newcommand{\seq}[1]{\left<#1\right>}
\newcommand{\norm}[1]{\left\Vert#1\right\Vert}
%----------------------- Macros and Definitions--------------------------
\newtheorem{thm}{Theorem}[section]
\newtheorem{pro}{Problem}
\newtheorem{cor}[thm]{Corollary}
\newtheorem{lem}[thm]{Lemma}
\newtheorem{prop}[thm]{Proposition}
\theoremstyle{definition}
\newtheorem{defn}[thm]{Definition}
\theoremstyle{remark}
\newtheorem{rem}[thm]{Remark}



%%% FILL THIS OUT
\newcommand{\studentname}{�̴뼮}
\newcommand{\idnumber}{20170450}
% You should write your "Student ID number" and your name here.
% You can write your name in Korean.
\newcommand{\exerciseset}{Assignment 7}
% If it is jth homework, then substitute n by j.
%%% END

\renewcommand{\theenumi}{\bf \Alph{enumi}}

\fancypagestyle{plain}{}
\pagestyle{fancy}
\fancyhf{}
\fancyhead[LO,RE]{\sffamily\bfseries\large MAS478 Discrete Geometry}
\fancyhead[RO,LE]{\sffamily\bfseries\large \idnumber:  \studentname}
\fancyfoot[RO,LE]{\sffamily\bfseries\thepage}
\renewcommand{\headrulewidth}{1pt}
\renewcommand{\footrulewidth}{1pt}

\graphicspath{{figures/}}

%-------------------------------- Title ----------------------------------

\title{MAS478 \exerciseset}
\author{\studentname \qquad ID: \idnumber}
\date{}

%--------------------------------- Text ----------------------------------

\begin{document}
\maketitle
